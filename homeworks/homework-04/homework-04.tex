\documentclass{article}

\usepackage{fancyhdr}
\usepackage{extramarks}
\usepackage{amsmath}
\usepackage{amsthm}
\usepackage{amsfonts}
\usepackage{tikz}
\usepackage[plain]{algorithm}
\usepackage{algpseudocode}
\usepackage{listings}
\usepackage{xcolor}
\usepackage[english]{babel}
\usepackage[T1]{fontenc}
\usepackage{lmodern,mathrsfs}
\usepackage{xparse}
\usepackage[inline,shortlabels]{enumitem}
\setlist{topsep=2pt,itemsep=2pt,parsep=0pt,partopsep=0pt}
\usepackage[dvipsnames]{xcolor}
\usepackage[utf8]{inputenc}
\usepackage[a4paper,top=0.5in,bottom=0.2in,left=0.5in,right=0.5in,footskip=0.3in,includefoot]{geometry}
\usepackage[most]{tcolorbox}
\tcbuselibrary{minted} % tcolorbox minted library, required to use the "minted" tcb listing engine (this library is not loaded by the option [most])
\usepackage{minted} % Allows input of raw code, such as Python code
% \usepackage[colorlinks]{hyperref}


\usetikzlibrary{automata,positioning}

\tcbset{
    pythoncodebox/.style={
        enhanced jigsaw,breakable,
        colback=gray!10,colframe=gray!20!black,
        boxrule=1pt,top=2pt,bottom=2pt,left=2pt,right=2pt,
        sharp corners,before skip=10pt,after skip=10pt,
        attach boxed title to top left,
        boxed title style={empty,
            top=0pt,bottom=0pt,left=2pt,right=2pt,
            interior code={\fill[fill=tcbcolframe] (frame.south west)
                --([yshift=-4pt]frame.north west)
                to[out=90,in=180] ([xshift=4pt]frame.north west)
                --([xshift=-8pt]frame.north east)
                to[out=0,in=180] ([xshift=16pt]frame.south east)
                --cycle;
            }
        },
        title={#1}, % Argument of pythoncodebox specifies the title
        fonttitle=\sffamily\bfseries
    },
    pythoncodebox/.default={}, % Default is No title
    %%% Starred version has no frame %%%
    pythoncodebox*/.style={
        enhanced jigsaw,breakable,
        colback=gray!10,coltitle=gray!20!black,colbacktitle=tcbcolback,
        frame hidden,
        top=2pt,bottom=2pt,left=2pt,right=2pt,
        sharp corners,before skip=10pt,after skip=10pt,
        attach boxed title to top text left={yshift=-1mm},
        boxed title style={empty,
            top=0pt,bottom=0pt,left=2pt,right=2pt,
            interior code={\fill[fill=tcbcolback] (interior.south west)
                --([yshift=-4pt]interior.north west)
                to[out=90,in=180] ([xshift=4pt]interior.north west)
                --([xshift=-8pt]interior.north east)
                to[out=0,in=180] ([xshift=16pt]interior.south east)
                --cycle;
            }
        },
        title={#1}, % Argument of pythoncodebox specifies the title
        fonttitle=\sffamily\bfseries
    },
    pythoncodebox*/.default={}, % Default is No title
}

% Custom tcolorbox for Python code (not the code itself, just the box it appears in)
\newtcolorbox{pythonbox}[1][]{pythoncodebox=#1}
\newtcolorbox{pythonbox*}[1][]{pythoncodebox*=#1} % Starred version has no frame

% Custom minted environment for Python code, NOT using tcolorbox
\newminted{python}{autogobble,breaklines,mathescape}

% Custom tcblisting environment for Python code, using the "minted" tcb listing engine
% Adapted from https://tex.stackexchange.com/a/402096
\NewTCBListing{python}{ !O{} !D(){} !G{} }{
    listing engine=minted,
    listing only,
    pythoncodebox={#1}, % First argument specifies the title (if any)
    minted language=python,
    minted options/.expanded={
        autogobble,breaklines,mathescape,
        #2 % Second argument, delimited by (), denotes options for the minted environment
    },
    #3 % Third argument, delimited by {}, denotes options for the tcolorbox
}


% Basic Document Settings
\topmargin=-0.45in
\evensidemargin=0in
\oddsidemargin=0in
\textwidth=6.5in
\textheight=9.0in
\headsep=0.25in
\linespread{1.1}

\pagestyle{fancy}
\lhead{\hmwkAuthorName}
\chead{\hmwkClass\ (\hmwkClassInstructor): \hmwkTitle}
\rhead{\firstxmark}
\lfoot{\lastxmark}
\cfoot{\thepage}
\renewcommand\headrulewidth{0.4pt}
\renewcommand\footrulewidth{0.4pt}
\setlength\parindent{0pt}

%
% Create Problem Sections
%

\newcommand{\enterProblemHeader}[1]{
	\nobreak\extramarks{}{Problem \arabic{#1} continued on next page\ldots}\nobreak{}
	\nobreak\extramarks{Problem \arabic{#1} (continued)}{Problem \arabic{#1} continued on next page\ldots}\nobreak{}
}

\newcommand{\exitProblemHeader}[1]{
	\nobreak\extramarks{Problem \arabic{#1} (continued)}{Problem \arabic{#1} continued on next page\ldots}\nobreak{}
	\stepcounter{#1}
	\nobreak\extramarks{Problem \arabic{#1}}{}\nobreak{}
}

\setcounter{secnumdepth}{0}
\newcounter{partCounter}
\newcounter{homeworkProblemCounter}
\setcounter{homeworkProblemCounter}{1}
\nobreak\extramarks{Problem \arabic{homeworkProblemCounter}}{}\nobreak{}

%
% Homework Problem Environment
%
% This environment takes an optional argument. When given, it will adjust the
% problem counter. This is useful for when the problems given for your
% assignment aren't sequential. See the last 3 problems of this template for an
% example.
%
\newenvironment{homeworkProblem}[1][-1]{
	\ifnum#1>0
		\setcounter{homeworkProblemCounter}{#1}
	\fi
	\section{Problem \arabic{homeworkProblemCounter}}
	\setcounter{partCounter}{1}
	\enterProblemHeader{homeworkProblemCounter}
}{
	\exitProblemHeader{homeworkProblemCounter}
}

%
% Homework Details
%   - Title
%   - Due date
%   - Class
%   - Section/Time
%   - Instructor
%   - Author
%

\newcommand{\hmwkTitle}{Problem Set\ \#3}
\newcommand{\hmwkDueDate}{Jun 5, 2025}
\newcommand{\hmwkClass}{ECON 124}
\newcommand{\hmwkClassInstructor}{Dr. Deniz Baglan}
\newcommand{\hmwkAuthorName}{\textbf{Alejandro Ouslan}}

%
% Title Page
%

\title{
	\vspace{2in}
	\textmd{\textbf{\hmwkClass:\ \hmwkTitle}}\\
	\normalsize\vspace{0.1in}\small{Due\ on\ \hmwkDueDate}\\
	\vspace{0.1in}\large{\textit{\hmwkClassInstructor}}
	\vspace{3in}
}

\author{\hmwkAuthorName}
\date{}

\renewcommand{\part}[1]{\textbf{\large Part \Alph{partCounter}}\stepcounter{partCounter}\\}

%
% Various Helper Commands
%

% Useful for algorithms
\newcommand{\alg}[1]{\textsc{\bfseries \footnotesize #1}}

% For derivatives
\newcommand{\deriv}[1]{\frac{\mathrm{d}}{\mathrm{d}x} (#1)}

% For partial derivatives
\newcommand{\pderiv}[2]{\frac{\partial}{\partial #1} (#2)}

% Integral dx
\newcommand{\dx}{\mathrm{d}x}

% Alias for the Solution section header
\newcommand{\solution}{\textbf{\large Solution}}

% Probability commands: Expectation, Variance, Covariance, Bias
\newcommand{\E}{\mathrm{E}}
\newcommand{\Var}{\mathrm{Var}}
\newcommand{\Cov}{\mathrm{Cov}}
\newcommand{\Bias}{\mathrm{Bias}}

\begin{document}

\maketitle

\pagebreak

% Homework problem 1
\begin{homeworkProblem}
	The file entitle \textbf{MONEYDEM.XLS} contains quarerly values of seasonally adjusted U.S.3-month
	(TB3mo) and 1-year (TB1yr) treasury bill rates. Each series is measured over the period 1959:Q3 to
	2001:Q1.
	\begin{enumerate}[(a)]
		\item Plot the time series of each series seperately. Does earch series appear to have a constant mean and
		      variance over time?
		\item Plot each time series on the same figure. What can you say about the relationship between the two series?
		\item Use OLS to estimatethe relatiohnship between long-jterm and short-term interes rates as :
		      $$
			      TB1yr_t = \alpha + \beta TB3mo_t + \epsilon_t
		      $$
		\item What does the estimate of $\beta$ tell you about the relationship between long-run and short-run
		      interes rates?
		\item Test the null that $\beta = 1$. Is this result in accortdance with macroeconomic theory?
		\item Plot the residuals from the regression in part (c) vs. TB3mo. Do you observe any pattern?
		\item Use the White Test to test for the presence of heteroskedasticity.
		\item Estimate the mmodel again, but calculate the robust (White) standard errors.
		\item What happens to the coefficients of the model in part (h) relative to part (c)? What happens
		      to the standart errors of the model in part (h) relative to part (c)? Why?
		\item Create a dummy variable that is equal to 1 when TB3mo is in excess of 10.00 and zero otherwise.
		      Include this variable in the regression model as
		      $$
			      TB1yr_t = \alpha + \beta TB3mo_t + \delta D_t + epsilon_t
		      $$
		      and run OLS
		\item Test the null that the dummy variable is relevant in part (j).
		      What happens to the fit of the model in part (j) relative to part (c)?
		\item What happens to th efit of the model in part (j) relative to part (c)?
	\end{enumerate}
\end{homeworkProblem}

% Homework problem 2
\begin{homeworkProblem}
	The file entiled SIM\_2.XLS contains simulated data sets. The series Y1, contains (T=100) values of
	a simulated AR(1) process. Use this series to perferm the following task.
	\begin{enumerate}[(a)]
		\item Plot the sequence against time. Does the series appear to be stationary?
		\item Plot the SCF and PACF.
		\item Estimate the AR(1), AR(2), ARMA(1,1), ARMA(1,4), and ARMA(2,1) models with intercepts.
		\item Estimate the series as both and AR(2) and ARMA (1,1) process without an intercept.
		\item Use $\bar{R}^2$, AIC and SC to choose the best single model over parts (c) and (d).
		\item Are you surprised by the result from part (e)? Why or why not?
		\item Using your idel model, plot the ACF and PACF of the residuals. Do they appear to be white noise?
	\end{enumerate}
\end{homeworkProblem}

% Homework problem 3
\begin{homeworkProblem}
	The file QUARTERLY.XLS contains the quarterly values of the CPI that have not been seasonally adjusted
	(CPINSA). The series is over the period 1960:Q1 to 2008:Q1.
	\begin{enumerate}[(a)]
		\item Plot the CPINSA sequence against time. Does the series appear to be stationary?
		\item Plot the ACF and PACF of CPINSA.
		\item Create the growth rate series $log(CPINSA_{t}/ CPINSA_{t-1})$ and plot this series against time.
		      Does the series appear to be stationary?
		\item Plot the ACF and PACF of $log(CPINSA_{t}/ CPINSA_{t-1})$
		\item Create the growth rate series $log(CPINSA_{t}/ CPINSA_{t-4})$ and plot this series against time.
		      Does the series appear to be stationary?
		\item Plot the ACF and PACF of $log(CPINSA_{t}/ CPINSA_{t-4})$
		\item Use the ACF and PACF from part (f) and estimate a tentative model. Try several other alternative
		      models.
		\item Use $\bar{R}^2$, AIC and SC to choose the best model from part (g).
		\item Instead of seasonally differing the series, regress $log(CPINSA_{t}/ CPINSA_{t-1})$ on (three) dummy variables to
		      controll for seasonality.
		\item Plot the residuals in part (i) versus time. Does this series appear to be stationary?
		\item Plot the ACF and PACF for the residuals in part (i). What do you conclude here?
	\end{enumerate}

\end{homeworkProblem}

% Homework problem 4
\begin{homeworkProblem}
	The file QUARTERLY.XLS contains U.S. interest rate over the period 1960:Q1 to 2008:Q1. Our
	goal here is to estimate a quarerly model of spread between a long-term and a short-term interest rate.
	Specifficaly, the interest rate spread (s) can be formed as the difference between the interest rate on
	a 10-year U.S. goverment bonds (r10) and the rate on a three-month treasury bills (Tbill) as
	$$
		s_t = r10_t - Tbill_t
	$$
	\begin{enumerate}[(a)]
		\item Plot $s_t$ against time. Does the series appear to be stationary?
		\item Plot the ACF and PACF of the time series. What do you conclude?
		\item Estimate an AR(2) model for $s_t$.
		\item Look at the ACF and PACF of the residuals from the regression in part (c). What do the
		      Ljung-Box Q-statistics say about autocorrelation in the residuals?
		\item  Estimate an AR(7) model for $s_t$.
		\item Look at the ACF and PACF of the residuals from the regression in part (e). What do the
		      Ljung-Box Q-statistics say anout autocorrelation in the residuals?
		\item Which model appears to perform better in terms of goodness-of-fit measures and diagnostic checks?
		\item Estimate both the AR(2) and AR(7) models over the period 1960:Q1 to 2005:Q3. Obtain the
		      one-step-ahead forecast and the one-step-ahead forcast error
		      $$
			      \hat{e}_{t+1} = y_{t_1} - \hat{y}_{t_1|t}
		      $$
		      for 2005:Q4($\hat{y}_{t+1|t})$ and compare that the true value of $s_t$ in 2005:Q4($y_{t+1}$). Which model has
		      the smaller forecast error? Hint: this may be easier to compute in Excel (after etimation).
		\item Estimate a ten-step-ahead forcast for each model as in part (h). Which model has the smallest mean
		      square forecast error
		      \[
			      \begin{split}
				      MSE & = \frac{1}{10}\sum_{i=1}^{10}\hat{e}_{t+1}^2               \\
				          & = \frac{1}{10}\sum_{i=1}^{10}(y_{t+1} - \hat{y}_{t+i|t})^2
			      \end{split}
		      \]
		      Which model performs better? Is this surprising?
	\end{enumerate}
\end{homeworkProblem}


\end{document}
