\documentclass{article}

\usepackage{fancyhdr}
\usepackage{extramarks}
\usepackage{amsmath}
\usepackage{amsthm}
\usepackage{amsfonts}
\usepackage{tikz}
\usepackage[plain]{algorithm}
\usepackage{algpseudocode}
\usepackage{listings}
\usepackage{xcolor}
\usepackage[english]{babel}
\usepackage[T1]{fontenc}
\usepackage{lmodern,mathrsfs}
\usepackage{xparse}
\usepackage[inline,shortlabels]{enumitem}
\setlist{topsep=2pt,itemsep=2pt,parsep=0pt,partopsep=0pt}
\usepackage[dvipsnames]{xcolor}
\usepackage[utf8]{inputenc}
\usepackage[a4paper,top=0.5in,bottom=0.2in,left=0.5in,right=0.5in,footskip=0.3in,includefoot]{geometry}
\usepackage[most]{tcolorbox}
\tcbuselibrary{minted} % tcolorbox minted library, required to use the "minted" tcb listing engine (this library is not loaded by the option [most])
\usepackage{minted} % Allows input of raw code, such as Python code
% \usepackage[colorlinks]{hyperref}


\usetikzlibrary{automata,positioning}

\tcbset{
    pythoncodebox/.style={
        enhanced jigsaw,breakable,
        colback=gray!10,colframe=gray!20!black,
        boxrule=1pt,top=2pt,bottom=2pt,left=2pt,right=2pt,
        sharp corners,before skip=10pt,after skip=10pt,
        attach boxed title to top left,
        boxed title style={empty,
            top=0pt,bottom=0pt,left=2pt,right=2pt,
            interior code={\fill[fill=tcbcolframe] (frame.south west)
                --([yshift=-4pt]frame.north west)
                to[out=90,in=180] ([xshift=4pt]frame.north west)
                --([xshift=-8pt]frame.north east)
                to[out=0,in=180] ([xshift=16pt]frame.south east)
                --cycle;
            }
        },
        title={#1}, % Argument of pythoncodebox specifies the title
        fonttitle=\sffamily\bfseries
    },
    pythoncodebox/.default={}, % Default is No title
    %%% Starred version has no frame %%%
    pythoncodebox*/.style={
        enhanced jigsaw,breakable,
        colback=gray!10,coltitle=gray!20!black,colbacktitle=tcbcolback,
        frame hidden,
        top=2pt,bottom=2pt,left=2pt,right=2pt,
        sharp corners,before skip=10pt,after skip=10pt,
        attach boxed title to top text left={yshift=-1mm},
        boxed title style={empty,
            top=0pt,bottom=0pt,left=2pt,right=2pt,
            interior code={\fill[fill=tcbcolback] (interior.south west)
                --([yshift=-4pt]interior.north west)
                to[out=90,in=180] ([xshift=4pt]interior.north west)
                --([xshift=-8pt]interior.north east)
                to[out=0,in=180] ([xshift=16pt]interior.south east)
                --cycle;
            }
        },
        title={#1}, % Argument of pythoncodebox specifies the title
        fonttitle=\sffamily\bfseries
    },
    pythoncodebox*/.default={}, % Default is No title
}

% Custom tcolorbox for Python code (not the code itself, just the box it appears in)
\newtcolorbox{pythonbox}[1][]{pythoncodebox=#1}
\newtcolorbox{pythonbox*}[1][]{pythoncodebox*=#1} % Starred version has no frame

% Custom minted environment for Python code, NOT using tcolorbox
\newminted{python}{autogobble,breaklines,mathescape}

% Custom tcblisting environment for Python code, using the "minted" tcb listing engine
% Adapted from https://tex.stackexchange.com/a/402096
\NewTCBListing{python}{ !O{} !D(){} !G{} }{
    listing engine=minted,
    listing only,
    pythoncodebox={#1}, % First argument specifies the title (if any)
    minted language=python,
    minted options/.expanded={
        autogobble,breaklines,mathescape,
        #2 % Second argument, delimited by (), denotes options for the minted environment
    },
    #3 % Third argument, delimited by {}, denotes options for the tcolorbox
}


% Basic Document Settings
\topmargin=-0.45in
\evensidemargin=0in
\oddsidemargin=0in
\textwidth=6.5in
\textheight=9.0in
\headsep=0.25in
\linespread{1.1}

\pagestyle{fancy}
\lhead{\hmwkAuthorName}
\chead{\hmwkClass\ (\hmwkClassInstructor): \hmwkTitle}
\rhead{\firstxmark}
\lfoot{\lastxmark}
\cfoot{\thepage}
\renewcommand\headrulewidth{0.4pt}
\renewcommand\footrulewidth{0.4pt}
\setlength\parindent{0pt}

%
% Create Problem Sections
%

\newcommand{\enterProblemHeader}[1]{
	\nobreak\extramarks{}{Problem \arabic{#1} continued on next page\ldots}\nobreak{}
	\nobreak\extramarks{Problem \arabic{#1} (continued)}{Problem \arabic{#1} continued on next page\ldots}\nobreak{}
}

\newcommand{\exitProblemHeader}[1]{
	\nobreak\extramarks{Problem \arabic{#1} (continued)}{Problem \arabic{#1} continued on next page\ldots}\nobreak{}
	\stepcounter{#1}
	\nobreak\extramarks{Problem \arabic{#1}}{}\nobreak{}
}

\setcounter{secnumdepth}{0}
\newcounter{partCounter}
\newcounter{homeworkProblemCounter}
\setcounter{homeworkProblemCounter}{1}
\nobreak\extramarks{Problem \arabic{homeworkProblemCounter}}{}\nobreak{}

%
% Homework Problem Environment
%
% This environment takes an optional argument. When given, it will adjust the
% problem counter. This is useful for when the problems given for your
% assignment aren't sequential. See the last 3 problems of this template for an
% example.
%
\newenvironment{homeworkProblem}[1][-1]{
	\ifnum#1>0
		\setcounter{homeworkProblemCounter}{#1}
	\fi
	\section{Problem \arabic{homeworkProblemCounter}}
	\setcounter{partCounter}{1}
	\enterProblemHeader{homeworkProblemCounter}
}{
	\exitProblemHeader{homeworkProblemCounter}
}

%
% Homework Details
%   - Title
%   - Due date
%   - Class
%   - Section/Time
%   - Instructor
%   - Author
%

\newcommand{\hmwkTitle}{Problem Set\ \#3}
\newcommand{\hmwkDueDate}{Jun 5, 2025}
\newcommand{\hmwkClass}{ECON 124}
\newcommand{\hmwkClassInstructor}{Dr. Deniz Baglan}
\newcommand{\hmwkAuthorName}{\textbf{Alejandro Ouslan}}

%
% Title Page
%

\title{
	\vspace{2in}
	\textmd{\textbf{\hmwkClass:\ \hmwkTitle}}\\
	\normalsize\vspace{0.1in}\small{Due\ on\ \hmwkDueDate}\\
	\vspace{0.1in}\large{\textit{\hmwkClassInstructor}}
	\vspace{3in}
}

\author{\hmwkAuthorName}
\date{}

\renewcommand{\part}[1]{\textbf{\large Part \Alph{partCounter}}\stepcounter{partCounter}\\}

%
% Various Helper Commands
%

% Useful for algorithms
\newcommand{\alg}[1]{\textsc{\bfseries \footnotesize #1}}

% For derivatives
\newcommand{\deriv}[1]{\frac{\mathrm{d}}{\mathrm{d}x} (#1)}

% For partial derivatives
\newcommand{\pderiv}[2]{\frac{\partial}{\partial #1} (#2)}

% Integral dx
\newcommand{\dx}{\mathrm{d}x}

% Alias for the Solution section header
\newcommand{\solution}{\textbf{\large Solution}}

% Probability commands: Expectation, Variance, Covariance, Bias
\newcommand{\E}{\mathrm{E}}
\newcommand{\Var}{\mathrm{Var}}
\newcommand{\Cov}{\mathrm{Cov}}
\newcommand{\Bias}{\mathrm{Bias}}

\begin{document}

\maketitle

\pagebreak

% Homework problem 1
\begin{homeworkProblem}
	The following sample moments for $x = [1,x_1,x_2,x_3]$ were computed from $100$ observations
	produced using a random number generator:
	$$
		X'X \begin{bmatrix}
			100 & 123 & 96  & 109 \\
			123 & 252 & 125 & 189 \\
			96  & 125 & 167 & 146 \\
			109 & 189 & 146 & 168
		\end{bmatrix} \quad X'y = \begin{bmatrix} 460 \\ 810 \\ 615 \\ 712\end{bmatrix} \quad y'y = 3924
	$$
	The true model underlying these data is $ y = x_1 + x_2 + x_3 + \epsilon$.
	\begin{enumerate}
		\item Compute the simple correlation among the regressors.
		\item Compute the ordinary least squares coefficients in the regression of $y$ on a constant, $x_1$, $x_2$, and $x_3$.
		\item Compute the ordinary least squares coefficients in the regression of $y$ on a
		      constant $x_1$ and $x_2$, on a constant, $X_1$ and $x_3$, and on a constant, $x_2$ and $x_3$.
		\item Compute the variance inflation factor associated with each variable.
		\item The regressors are obviously badly collinear, Which is the problem variable? Explain
	\end{enumerate}
\end{homeworkProblem}

% Homework problem 2
\begin{homeworkProblem}
	A multiple regression of $y$ on a constant $x_1$ and $x_2$ produces the following resutls:
	$$
		\hat{y}= 4 + 0.4x_1 + 0.9x_2 \quad R^2 = \frac{8}{60} \quad e'e = 520, \quad n = 29,
	$$
	$$
		X'X= \begin{bmatrix}
			29 & 0 & 0 \\ 0 & 50 & 10 \\ 0 & 10 & 80
		\end{bmatrix}
	$$
	Test the hypothesis that the two slopes sum to 1
\end{homeworkProblem}

% Homework problem 3
\begin{homeworkProblem}
	The application in Chapter 3 used 15 of the 19,919 observations in Koop and Tobias's
	(2004) study of the relationship between wages and education, ability, and family
	characteristics. (See Appendix Table F3.2.) We will use the full data set for this
	exercise. The data may be downloaded from the \textit{Journal of Applied Econometrics}
	data archive at \textbf{link}. The data file is in two parts. The fist file contains
	the panel of 19,919 obseervations on variables:

	To create the data set for this exercise, it is necessary to merge these two data
	files. The \textit{i}th observations in the first file will be replicated $T_i$ times for the
	set of $T_i$ observations in the first file. The \textit{person id} variable indicates which rows
	must contain the data from the second file. (How this preparation is carried out will)
	vary from one computer package to another.) (\textit{Note:} We are not attempting to replicate
	the data set.) Let
	$$X_1 = [constant, education, experience, ability]$$
	$$X_2 = [mother's education, father's education,broken home, number of siblings]$$

	\begin{enumerate}
		\item compute the full regression of $(\ln{wage} \sim X_1)$ and $(\ln{wage} \sim X_2)$
		\item Use the $F$ test to test the hypothesis that all coefficients except the constant
		      term are zero.
		\item Use the $F$ statistic to test the joint hypothesis that the coefficient on the four
		      household variables in $X_2$ are zero
		\item Use a Wald test to carry out the test in part $c$.
	\end{enumerate}
\end{homeworkProblem}

% Homework problem 4
\begin{homeworkProblem}
	In a paper in 1963, Mare Nerlove analyzed a cost function for 145 American electric companies.
	The attached data file, contains the data and the description file. Nerlove was interested in
	estimating a cost function: $TC = f(Q,PL, PF,PK)$.
	\begin{enumerate}
		\item First estimate an unrestricted Cobb-Douglas specification
		      $$
			      \log{TC}_i = \beta_1 + \beta_2 \log{Q_i} + \beta_3 \log{PL_i} + \beta_4 \log{PK_i} + \beta_5 \log{PF_i} + \epsilon_i
		      $$
		      Report parameter estimates and standard errors.
		\item What is the economic meaning of the restriction $H_0:\beta_3 + \beta_4 + \beta_5 = 1$?
		\item Estimate the regression in (a) by constrained least squares $\beta_3 + \beta_4 + \beta_5 = 1$.
		      Report your parameter estimates and standard errors.
		\item Test $H_0:\beta_3 + \beta_4 + \beta_5 = 1$ using a Wald statistic.
	\end{enumerate}
\end{homeworkProblem}

\begin{homeworkProblem}
	Replicate Example 7.12 income elasticity of credit card expenditures in Green's
	textbook.the data set can be downloaded form the link below:
\end{homeworkProblem}

\end{document}
